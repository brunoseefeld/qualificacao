\documentclass{article}
\usepackage{graphicx} % Required for inserting images
\usepackage{amsmath}
\usepackage{amsfonts}
\usepackage[]{dsfont}




\title{Ementa de qualificação}
\author{Bruno Seefeld\\
Orientador: Alejandro Kocsard}
\date{Agosto 2024}

\begin{document}

\maketitle


\section{Dinâmica Hiperbólica}

\begin{itemize}
    \item Classificação de diffeomorfismos estruturalmente estáveis do intervalo
    \item Número de rotação de homeomorfismo do círculo
    \item Classificação de diffeomorfismos estruturalmente estáveis do círculo
    \item Conjuntos hiperbólicos
    \item Cone field criteria
    \item Teorema de Grobman-Hartman
    \item Existência de variedades estáveis e instáveis para pontos periódicos hiperbólicos
    \item Shaddowing e Anosov closing lemma
    \item Decomposição espectral
    \item Espectro de Mather




\end{itemize}


\section{Teoria Ergódica Diferenciável}
\begin{itemize}
    \item Teorema de Recorrência de Poincaré
    \item Teorema de Kac 
    \item Teorema de Prohorov
    \item Teorema da existência de medidas invariantes para funções contínuas em um compacto.
    \item Teorema ergódico de von-Neumann para isometrias em $L_2(X,\mu)$.
    \item Teorema ergódico maximal
    \item Teorema ergódico de Birkhoff
    \item Ergodicidade e mixing
    \item Caracterização espectral de ergodicidade e mixing.
    \item Entropia e fórmula de Komolgorov-Sinai
\end{itemize}


\section{Topologia Diferencial}
\begin{itemize}
    \item Teorema de Sard 
    \item Whittney embedding 
    \item Topologia $C^k$ 
    \item Espaço de jatos 
   
    \item Teorema de Transversalidade de Thom para variedades sem bordo
    \item Lema de Morse e Teorema de Morse para variedades sem bordo
   
\end{itemize}


\section{Teoria espectral}

\begin{itemize}
    \item Teorema de Gelfand da existência do espectro para álgebras de Banach
    \item Caracteres de álgebras de Banach
    \item Teorema de Representação de Gelfand para álgebras de Banach abelianas
    \item Teorema de Gelfand da caracterização de álgebras C* abelianas.
    \item Spectral mapping theorem para álgebras C*.
    \item Teorema espectral para operadores normais em $B(H)$
    \item Cálculo funcional Boreliano para os respectivos operadores
    \item Subálgebras hereditárias de álgebras C*
    \item Teorema de representação de Gelfand-Naimark-Segal
    \item Teorema do comutante duplo para álgebras de von-Neumann. 


\end{itemize}


\end{document}